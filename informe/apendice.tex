\section{Apéndices}
	\subsection{Apéndice I: generación y análisis de datos}
	Para poder analizar las complejidades de los algoritmos propuestos, se utilizaron las siguientes herramientas para generar mediciones y graficar datos:

	\begin{itemize}
		\item Un generador de grafos aleatorios con \texttt{std::random} (parte de la biblioteca estándar de C++11).

		Dependiendo los distintos requerimientos de los diversos problemas, tuvimos que crear distintos generadores de grafos. En particular:

		\begin{enumerate}
			\item Un generador de grafos completos, utilizado para el problema 3 y como base para otros generadores
			\item Un generador de árboles, usando grafos completos como base y un algoritmo basado en Kruskal para obtener el arbol generador
			\item Un generador de grafos conexos, utilizado en el problema 1, partiendo de un grafo y un arbol generador y agregando ejes adicionales
			\item Un generador de grafos no necesariamente conexos, utilizado para el problema 2.
		\end{enumerate}

		Los generadores utilizan generadores de distribuciones uniformes provistas por la \texttt{std} de C++.

		Cada uno de los grafos generados se utilizó multiples veces para testear varios aspectos de un mismo problema.

		\item Mediciones de tiempo con \texttt{std::chrono} (parte de la biblioteca estándar de C++11).

		Cada algoritmo fue probado 100 veces consecutivas con cada entrada, conservando solo el valor de tiempo menor para reducir el ruido por procesos ajenos al problema.

		La unidad de medición preferida fue microsegundos (\texttt{std::chrono::microseconds}, $seg \times 10^{-6}$).

		\item Graficado con \texttt{matplotlib.pyplot}, \texttt{pandas} y \texttt{seaborn} (con Python y Jupyter Notebook)

		Se utilizaron los DataFrames de Pandas para el manejo de datos (guardados en \texttt{.csv}), y las funciones de regresión de Seaborn para el graficado, en conjunto con matplotlib.

	\end{itemize}

	Los gráficos incluyen el coeficiente de correlación (o "r") de Pearson, al igual que un p-valor para la hipotesis nula que los valores pueden haber sido generados sin correlación real. Esto quiere decir, un R cercano a 1 con un p-valor mínimo indica que hay una fuerte correlación positiva entre los valores graficados; un p-valor elevado, por otro lado, invalida una posible correlación positiva o negativa.

	\subsection{Apéndice II: herramientas de compilación y testing}
	Durante el desarrollo se utilizaron las siguientes herramientas:

	\begin{itemize}
		\item CMake

		Se decidió utilizar CMake para la compilación por su simplicidad y compatibilidad con otras herramientras. Junto con el código se provee el archivo \texttt{CMakeLists.txt} para compilar el mismo.

		\item Google Test

		Para generar tests unitarios con datos reutilizables se usó Google Test. Dichos archivos eran importados por otro \texttt{CMakeLists.txt} y no están incluídos en la presente entrega del trabajo práctico.

		\item Namespace \texttt{Utils}

		Dentro de \texttt{Utils.h} se definió una función de logging que fue utilizada al programar para detectar errores y ver otros detalles del proceso. La función \texttt{log} sigue estando incluída en los algoritmos, pero su funcionalidad se encuentra apagada por un \texttt{\#define} y no debería generar ningún costo adicional (ya que usa printf por detrás y no genera el output salvo que sea necesario).
	\end{itemize}