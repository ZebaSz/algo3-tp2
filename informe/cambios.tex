\renewcommand\labelitemii{$\circ$}
\section{Informe de cambios}
	\subsection*{Cambios generales}

	\begin{itemize}
		\item Se agregaron gráficos representando los ejemplos provistos para cada problema
		
		\item Se realizaron correcciones ortográficas y de redacción, para facilitar la lectura y el entendimiento

		\item Los gráficos de complejidad fueron rehechos para explicitar más información, en particular:

		\begin{itemize}
			\item Coeficiente de Correlación de Pearson

			\item Funciones de las curva graficadas

			\item Constantes utilizadas al analizar una sola variables
		\end{itemize}

		\item Se volvieron a tomar algunas mediciones que tenían ruido excesivo
	\end{itemize}

	\subsection*{Ejercicio I: Delivery óptimo}

	\begin{itemize}
		\item Se realizó un cambio en el algoritmo de Dijkstra, para mejorar su complejidad. En lugar de utilizar una matriz de adyacencia, pasamos a usar una cola de prioridad implementada con Fibonacci Heap y listas de adyacencia, de forma que la complejidad de este algoritmo sea la más óptima posible.
		
	\end{itemize}

	\subsection*{Ejercicio II: Subsidiando el transporte}

		Este algoritmo tuvo que ser refactorizado, ya que contaba con errores y su demostración era difícil e insuficiente. El mismo fue modificado de la siguiente forma:
		
		\begin{itemize}
			\item Al principio se agrega un nodo universal al grafo; el mismo es utilizado como origen siempre al correr Bellman-Ford

			\item Ya no se verifica si el grafo es conexo o cuáles son sus componentes conexas; el nodo universal es agregado siempre

			\item En consecuencia de lo anterior, ya no se utiliza DisjointSet para tomar representantes de las componentes conexas existentes
		\end{itemize}

		Esto nos permite respetar la cota de complejidad, simplificar el algoritmo y su demostración de correctitud, y considerar casos particulares que anteriormente fallaban.

		Debido a estos cambios, las mediciones se tomaron nuevamente y todas las demostraciones y los análisis fueron reescritos.

	\subsection*{Ejercicio III: Reconfiguración de rutas}

	\begin{itemize}
		\item Se reorganizó la demostración, buscando dar una idea más intuitiva y aclarar las demostraciones que podían resultar confusas
		
		\item Se rehizo la cota temporal, extendiendo la explicación y mejorando su dialéctica para facilitar el entendimiento
				
		\item En la experimentación, adicionalmente, se analizó la complejidad en base a la cantidad de aristas del grafo; esto no utiliza mediciones distintas, los valores del eje X fueron transformados teniendo en cuenta que $m = \frac{n * (n - 1)}{2}$, y se utiliza una curva acorde

	\end{itemize}