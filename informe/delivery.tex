\section{Delivery óptimo}

\subsection{Descripción del problema}
Agregar ejemplos
\\
\par
\subsection{Desarrollo}
Dado este problema, podemos modelarlo utilizando grafos. Así, cada ciudad se representaría con un nodo, y cada una de las rutas que conecta dos ciudades, con una arista. Del mismo modo, el problema pasaría a ser alcanzar el camino mínimo de un nodo origen a un nodo destino, utilizando como máximo k aristas premium; y dado que las rutas son bidireccionales, podemos utilizar un grafo común no direccionado para esta representación.
\\
\par
A partir de esta representación, podemos poner el foco en las dificultades que trae consigo el problema. Sabemos que utilizando un algoritmo de camino mínimo, obtendríamos el camino más corto desde el nodo origen al nodo destino, pero nada sabríamos sobre cuantas rutas premium se estan utilizando. De este modo, si bien tenemos una aproximación buena del problema, tenemos que buscar la manera de poner en consideración las rutas premium y su uso. 
\\
\par

\subsection{Cota temporal}

\subsection{Experimentacion}



 



