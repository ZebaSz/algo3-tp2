\section{Delivery óptimo}

\subsection{Descripción del problema}
El problema plantea una provincia en la cual las ciudades están conectadas por dos tipos de rutas: las comunes y las premium. En ambos casos, las rutas son bidireccionales, conectan dos ciudades y tienen asociada una distancia no negativa.
\\
\par
A través de estas rutas, nuestra empresa busca transportar mercadería desde una ciudad origen a una ciudad destino, de manera que se recorra la menor distancia posible considerando que, por regulaciones provinciales, solo se puede pasar por k rutas premium en este recorrido (es decir, que la suma de distancias de las rutas utilizadas sea la menor posible, utilizando entre 0 y k rutas premium). La complejidad del algoritmo debe ser no peor que \textbf{O($n^2k^2$)} donde n es la cantidad de ciudades y k la máxima cantidad de rutas premium utilizables.
\\
\par
\textbf{Agregar ejemplos}
\\
\par
\subsection{Desarrollo}
Dado este problema, podemos modelarlo utilizando grafos. Así, cada ciudad se representaría con un nodo, y cada una de las rutas que conecta dos ciudades, con una arista. Del mismo modo, el problema pasaría a ser alcanzar el camino mínimo de un nodo origen a un nodo destino, utilizando como máximo k aristas premium; y dado que las rutas son bidireccionales, podemos utilizar un grafo común no dirigido para esta representación.
\\
\par
A partir de esta representación, podemos poner el foco en las dificultades que trae consigo el problema. Sabemos que utilizando un algoritmo de camino mínimo, obtendríamos el camino más corto desde el nodo origen al nodo destino, pero nada sabríamos sobre cuantas rutas premium se están utilizando. De este modo, si bien tenemos una aproximación buena del problema, tenemos que buscar la manera de poner en consideración las rutas premium y su uso. 
\\
\par
Un buen punto de inicio es considerar que, si bien tenemos un límite de rutas premium, podríamos encontrar un mejor camino que utilice una menor cantidad. Por ende, no alcanza con encontrar el camino mínimo que utilice k rutas premium, sino que debemos encontrar el mínimo de todos aquellos que utilicen hasta k premium. Es decir que, en lugar de plantearlo como un único problema, podríamos ver el ejercicio como la mejor solución de k subproblemas distintos, donde el i-ésimo subproblema nos da el camino mínimo utilizando exactamente $i$ rutas premium.
\\
\par
Por otro lado, aún buscando resolver solo uno de los subproblemas planteados, estaríamos frente a la posibilidad de alcanzar el límite de rutas antes de recorrer todas (y por ende, de estar tomando una decisión errónea). Debemos, entonces, buscar la manera de llegar al último nodo habiendo elegido las mejores rutas premium posibles, siendo que no haya forma de tomar una ruta premium distinta y a través de ella se consiga un camino de menor distancia.
\\
\par
Para enfrentar este problema, podríamos generar un grafo con $k$ níveles, donde cada nivel sea una copia exacta del grafo original, y el nodo J perteneciente al i-ésimo nivel represente que para llegar a ese nodo se utilizaron $i$ rutas premium desde el nodo origen. De este modo, pasamos a tener un grafo de $nk$ nodos, donde cualquiera de los caminos entre el nodo origen y los k nodos destinos pueden representar una respuesta válida.
\\
\par
Sin embargo, modelando el grafo de esta manera tendríamos problemas, puesto que nuestro grafo inicial era no dirigido y nuestra idea es que cada nivel represente la cantidad de rutas premium utilizadas hasta el momento. Por ende, no podría ocurrir que se pase de un nodo perteneciente del nivel $i+1$ a un nodo del nivel $i$, ya que esto implicaría que, al utilizar una ruta premium más, disminuyó el nivel (es decir que disminuyó la cantidad de rutas premium utilizadas, siguiendo la idea de nuestro modelo, lo cual sería absurdo). Por lo tanto, debemos modelar el nuevo grafo de manera que:
\begin{enumerate}
\item Debe haber un nivel por cada posible respuesta
\item Cada vez que se utiliza una ruta premium, se sube de nivel
\item Cada vez que se utiliza una ruta que no es premium, se mantiene el mismo nivel
\item Se deben mantener las conexiones del grafo original (debe ser una representación fiel)
\end{enumerate}

Dadas estas condiciones, tendremos un grafo con $k$ niveles, como habíamos establecido anteriormente, pero debido a que hay rutas que solo se pueden recorrer en una dirección, tendremos que usar un \underline{\textbf{grafo dirigido}}. En él, estableceremos dos tipos de relaciones, que aplicarán para todos los niveles del grafo.
\\
\par
La primer relación a definir será entre los nodos conectados por una arista común. No es dificil ver que, si en el grafo original había una arista común entre el nodo J y el nodo H de peso $P$, ahora habrá $k$ representaciones de esa misma arista, para los nodos $J_i$ y $H_i$ con i entre 0 y k, donde el peso de cada una de ellas será $P$. Al ser nuestro nuevo grafo un digrafo, y como establecimos que cuando no se usa una ruta premium se mantiene el mismo nivel, bien podría ocurrir que nuestro camino vaya de $J_i$ a $H_i$ o viceversa, por lo que utilizaremos una arista para cada una de estas posibilidades. Es decir que, para cada nivel i, habrá una arista dirigida que conecte $J_i$ con $H_i$ y una que conecte $H_i$ con $J_i$. De este modo, quedan definidas de manera correcta todas las aristas no premium pertenecientes al grafo inicial.
\\
\par
La segunda relación a definir será entre los nodos conectados por una arista premium. Al igual que en el otro caso, por cada arista premium habrá $k$ representaciones de esta arista. Sin embargo, como al tomar una ruta premium debemos aumentar el nivel, hay que establecer una relación entre el i-ésimo y el (i+i)-ésimo nivel. Así, si en el grafo inicial había una arista premium entre los nodos J y K, ahora habrá una arista dirigida de $J_i$ a $K_i+1$ y una de $K_i$ a $J_i+1$. Notemos que, al tener una arista no dirigida en el grafo inicial, debe ser posible recorrer la ruta en ambas direcciones.
\\
\par
De esta manera, definimos nuestro grafo dirigido de $k$ niveles de manera que represente correctamente el grafo inicial y, por consecuente, a nuestro problema. Con esta nueva representación, un grafo asi: quedaría asi: \textbf{INSERTAR IMAGENES PAGANAS}
\\
\par
Dado que ya tenemos un grafo que nos permite identificar la cantidad de rutas premium utilizadas por la manera en la que está caracterizada cada nodo, basta con conocer nuestro nodo origen y buscar el camino más corto hasta las $k$ posibles soluciones, para luego quedarnos con la mejor solución.
\\
\par
Por lo tanto, como sabemos que el nodo origen necesariamente estará situado en el primer nivel (es decir, que es único para nuestros propósitos), podemos utilizar un algoritmo que nos de el camino mínimo de un nodo a todos los demás y luego evaluar solo aquellos que nos den una respuesta válida a nuestro problema (un camino de origen a destino).
\\
\par
En consecuencia, podríamos usar tanto el algoritmo de Bellman-Ford como el de Dijkstra. Sin embargo, considerando que no tenemos aristas con pesos negativos, y que el uso de Dijkstra nos facilita el cumplimiento dela complejidad, nos quedaremos con este algoritmo para realizar la búsqueda.
\\
\par
Por lo tanto, con lo visto anteriormente, acabaríamos teniendo un algoritmo de este estilo:
\\
\par
\textbf{PSEUDOCODIGO}

\subsection{Cota temporal}
Crear la representación del grafo con niveles cuesta para cada una de las aristas $O(k)$, y como hay m aristas en total es $O(mk)$ y como m está acotado por $n^2$, podemos decir también que es $O(n^2*k)$. Después aplicamos Dijkstra que la implementación que llevamos a cabo es $O(n^2)$ pero nuestro grafo ya ni tiene $n$ nodos, sino que tiene $n*(k+1)$ osea que para nuestro caso la complejidad temporal en peor caso es de es $O((n*(k+1))^2)$ o lo que es lo mismo $O(n^2*k^2)$. Por último nos queda buscar el valor óptimo que lo logramos en $O(k)$ ya que hacemos k iteraciones de conseguir un elemento de un arreglo que es es $O(1)$. Resumiendo la complejidad temporal nos queda en $O( n^2*k + n^2*k^2 + k )$ que es lo mismo que $O(n^2*k^2)$ que se ajusta a lo que pedía el ejercicio.

\subsection{Experimentacion}

\pagebreak