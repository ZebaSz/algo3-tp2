\section{Delivery óptimo}

\subsection{Descripción del problema}
El problema plantea una provincia en la cual las ciudades están conectadas por dos tipos de rutas: las comunes y las premium. En ambos casos, las rutas son bidireccionales, conectan dos ciudades y tienen asociada una distancia no negativa.
\\
\par
A través de estas rutas, nuestra empresa busca transportar mercadería desde una ciudad origen a una ciudad destino, de manera que se recorra la menor distancia posible considerando que, por regulaciones provinciales, solo se puede pasar por k rutas premium en este recorrido (es decir, que la suma de distancias de las rutas utilizadas sea la menor posible, utilizando entre 0 y k rutas premium). La complejidad del algoritmo debe ser no peor que O(n2k2) donde n es la cantidad de ciudades y k la máxima cantidad de rutas premium utilizables.
\\
\par
\textbf{Agregar ejemplos}
\\
\par
\subsection{Desarrollo}
Dado este problema, podemos modelarlo utilizando grafos. Así, cada ciudad se representaría con un nodo, y cada una de las rutas que conecta dos ciudades, con una arista. Del mismo modo, el problema pasaría a ser alcanzar el camino mínimo de un nodo origen a un nodo destino, utilizando como máximo k aristas premium; y dado que las rutas son bidireccionales, podemos utilizar un grafo común no direccionado para esta representación.
\\
\par
A partir de esta representación, podemos poner el foco en las dificultades que trae consigo el problema. Sabemos que utilizando un algoritmo de camino mínimo, obtendríamos el camino más corto desde el nodo origen al nodo destino, pero nada sabríamos sobre cuantas rutas premium se estan utilizando. De este modo, si bien tenemos una aproximación buena del problema, tenemos que buscar la manera de poner en consideración las rutas premium y su uso. 
\\
\par

\subsection{Cota temporal}

\subsection{Experimentacion}

\pagebreak


 



