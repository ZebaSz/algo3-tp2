\section{Reconfiguración de rutas}

\subsection{Descripción del problema}
Este problema plantea otra provincia de Optilandia, cuyas ciudades están conectadas por rutas pero con ciertos problemas: algunas ciudades no están conectadas y otras poseen varias maneras para viajar entre ellas. Para solucionarlo, el gobierno hará obras en las rutas de modo que haya una y sólo una forma de llegar desde cualquier ciudad a otra, construyendo nuevas rutas o destruyendo rutas existentes.
\\
\par
Nos piden un algoritmo que, dadas las rutas y los costos de construcción y destrucción, nos diga que rutas hay que destruir y construir para satisfacer el problema de forma que gastemos lo menor posible. Como requisito, la complejidad de este algoritmo no puede ser peor que \textbf{O($n^2log(n)$)}, donde n es la cantidad de ciudades de la provincia.
\\
\par
\textbf{Agregar ejemplos}
\\
\par
\subsection{Desarrollo}

\subsection{Cota temporal}

\subsection{Experimentacion}

\pagebreak